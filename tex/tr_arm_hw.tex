% Technical review - kvm/arm
\subsection{KVM/ARM, Xen ARM}
Существует два похожих решения, использующие поддержку аппаратной виртуализации. ARM Virtualization Extension(VE) - расширения для аппаратной виртуализации ARM. VE есть только на процессорах нового поколения (Cortex A15), которые появились в начале 2013 года и еще не используются в коммерческих продуктах. ARM VE добавляет в процессор новый режим гипервизора (hypervisor, hyp). Все новые возможности доступны только из этого режима - отдельное преобразование страниц для гостя, виртуальный контроллер прерываний, копии некоторых регистров процессора.

\textbf{KVM/ARM} 

KVM(Kernel-based Virtual Machine)\cite{bib:kvm} - программное решение, обеспечивающее виртуализацию  в среде Linux. Компоненты ядра, необходимые для KVM, включены в основное ядро начиная с версии 2.6.20.


KVM ARM\cite{bib:kvm_arm} не требует изменять код гостевой ОС, требуется только версия ядра выше 2.6.20, собранная с KVM модулем и с драйверами virtio. Virtio - библиотека для виртуализации ввода/вывода, добавляет интерфейс виртуальных устройств в систему(например, блочные устройства, сетевые устройства). Таким образом, гостевая система виртуализуется аппаратно, только ввод/вывод перенаправляется через virtio в хост, где эмулируется через Qemu.

KVM ARM обладает хорошей производительностью, не требует модификации кода гостя. Может работать только с ОС на ядре Linux и требует поддержки аппаратной виртуализации(процессоры Cortex-A15). KVM хорошо работает в FastModel(программная эмуляция платформы), но на реальном процессоре ведет себя нестабильно.

\textbf{XEN ARM}

Проект Xen ARM\cite{bib:xen_arm} - это вариант Xen Hypervisor для архитектуры ARM. Требует запуска в hypervisor режиме. В качестве гостевой системы для Xen ARM выступает паравиртуализованный Linux - Dom0. Xen планирует внедрять это решение для серверов на базе ARM.




