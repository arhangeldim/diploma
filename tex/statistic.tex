% Сбор статистики
Анализ реального кода позволяет понять, какие инструкции исполняются часто, или, наоборот, не исполняются. Также можно проанализировать частые шаблоны кода. Например, ARM является load/store архитектурой, то есть все операции с данными работают с регистрами, а результат вычислений сохраняется в памяти. Также в коде часто используются условные операции и операции перехода. В коде, скомпилированном под ARM, могут встречаться инструкции из нескольких наборов, причем они могут использоваться совместно. При загрузке ядра Linux встречаются инструкции  ARM, Thumb, Thumb2. Каждый набор надо анализировать отдельно, потому что инструкции имеют различный формат.

Эмулятор позволяет собирать информацию о исполняемом коде, в том числе и количество инструкций, их тип. Вообще, на разных задачах паттерны кода могут различаться и результаты, собранные на одной задаче, могут отличаться от результатов другой задачи. Были проведены несколько экспериментов на различных наборах данных - код ядра Linux, утилиты командной строки ps, ls. Несмотря на разные задачи, результаты получились схожие. С наибольшей частотой встречаются инструкции работы с памятью - ldr, str; арифметические и логические операции - sub, add, lsl, mov, cmp; операции переходов - b. Эти инструкции составляют до 80-90\% всего кода. Можно сделать допущение, что такое распределение инструкций в коде верно для большинства задач.

Далее рассмотрены обобщенные результаты для наборов инструкций ARM, Thumb, Thumb2. Доля инструкций вычисляется как отношение количества конкретной инструкции к количеству всех инструкций. На графиках отражены данные по наиболее популярным инструкциям, инструкции не указанные на графике имеют долю менее 1\%. Количество инструкций набора ARM - 214.

\begin{figure}[h!]
    \center{\includegraphics[width=0.9\linewidth]{statistic_arm_percent}}
    \caption{Статистика инструкций ARM }
    \label{img:stat_arm_percent}
\end{figure}

\begin{figure}[h!]
    \center{\includegraphics[width=0.9\linewidth]{statistic_thumb_percent}}
    \caption{Статистика инструкций Thumb}
    \label{img:stat_thumb_num}
\end{figure}

\begin{figure}[h!]
    \center{\includegraphics[width=0.9\linewidth]{statistic_thumb2_percent}}
    \caption{Статистика инструкций Thumb2}
    \label{img:stat_thumb2_num}
\end{figure}


\begin{table}[h!] \label{tab:arm_top}
	\caption{\label{tab:stat_arm_top} ARM инструкции с наибольшим весом}
	\begin{center}
		\begin{tabular} {|c|c|c|}
			\hline
			Инструкция & Опкод & Маска \\
			\hline
			ldr & 0x04100000 & 0x0e500000 \\
			\hline
			str & 0x04000000 & 0x0e500000 \\
			\hline
			b   & 0x0a000000 & 0x0f000000 \\
			\hline
			add & 0x02800000 & 0x0fe00000 \\
			\hline
			cmp imm & 0x03500000 & 0x0ff00000 \\
			\hline
			sub & 0x02400000 & 0x0fe00000 \\
			\hline
			add & 0x00800000 & 0x0fe00000 \\
			\hline
			lsl & 0x01a00000 & 0x0fe00070 \\
			\hline
			ldrb & 0x04500000 & 0x0e500000 \\
			\hline
			cmp reg & 0x01500000 & 0x0ff00010 \\
			\hline
			rsb & 0x00600000 & 0x0fe00010 \\
			\hline
			mov & 0x03a00000 & 0x0fe00000\\
			\hline
		\end{tabular}
	\end{center}
\end{table}

\begin{table}[h!]
	\caption{\label{tab:stat_thumb_top} Thumb инструкции с наибольшим весом}
	\begin{center}
		\begin{tabular} {|c|c|c|}
			\hline
			Инструкция & Опкод & Маска \\
			\hline
			cmp & 0x4280 & 0xffc0 \\
			\hline
			add & 0x3000 & 0xf800 \\
			\hline
			b   & 0xd000 & 0xf000 \\
			\hline
			b & 0xe000 & 0xf800 \\
			\hline
			uxtb & 0xb2c0 & 0xffc0 \\
			\hline
			ldrb & 0x5c00 & 0xfc00 \\
			\hline
			strb & 0x5400 & 0xfc00 \\
			\hline
			add & 0x1800 & 0xfe00 \\
			\hline
			cmp & 0x4500 & 0xff00 \\
			\hline
		\end{tabular}
	\end{center}
\end{table}

Есть группа инструкций с высокой долей, а затем идут инструкции с незначительной долей и таких инструкций большинство. Причем популярные инструкции составляют около 80\% всех исполняемых инструкций. В соответствие с найденной частотой вхождения, каждой инструкции можно назначить вес, число пропорциональное частоте вхождения.


