% Введение :
%   Область исследования
%   Проблема
%   Актуальность
%   Конкретная проблема, решаемая в работе

Данная работа является исследованием в области виртуализации архитектуры ARM. Термин виртуализация означает различные методы представления вычислительных ресурсов(например, аппаратная платформа, операционная система, хранилище данных), которые дают преимущества перед их оригинальной конфигурацией. Виртуализация - это общая концепция для многих аспектов вычислений, более подробно типы виртуализации будут рассмотрены ниже.

Архитектура ARM - семейство RISC процессоров, разрабатываемых британской компанией ARM Limited. Большинство современных мобильных устройств - смартфонов и планшетов - используют процессоры ARM и этот показатель постоянно растет. Например, только за первый квартал 2013 года было выпущено 2,6 млрд. процессоров\cite{bib:arm_web}.

Процессоры ARM отличаются низкой стоимостью, малым энергопотреблением и тепловыделением, что позволят использовать их в мобильных устройствах и встраиваемых системах. ARM используется в качестве процессора Apple iPad, Apple iPhone, Microsoft Surface, Samsung Galaxy. Также ARM-чипы используются в SOC системах Raspberry Pi, PandaBoard и других.

В отношении мобильных устройств возникает несколько проблем. Сейчас развивается технология BYOD(Bring your own device) - использование одного устройства для личных и корпоративных целей, при этом встает проблема изоляции двух окружений, с одной стороны безопасность данных компании, а с другой стороны приватность данных владельца. Обычно у одного владельца есть несколько мобильных устройств, например, телефон и планшет, тогда возникает задача миграции данных между устройствами, также при замене устройства было бы удобно переносить все данные и настройки со старого устройства на новое. Еще одна проблема - непереносимость приложений между мобильными операционными системами, виртуализация могла бы решить эту проблему, создав окружение, в котором можно запускать приложения, написанные для другой ОС.

Также, в 2012 году были представлены сервера на базе ARM, это означает возможность применения виртуализации для более эффективного управления ресурсами.

Сейчас существует несколько решений для виртуализации архитектуры ARM. Qemu полностью виртуализует платформу, Xen ARM и KVM ARM используют паравиртуализацию и аппаратную виртуализацию. Подход Qemu универсален, но требует полной эмуляции платформы, а значит обладает невысокой производительностью, Xen ARM имеет высокую скорость работы, но при этом требуется вносить изменения в гостевую ОС, KVM ARM использует аппаратную виртуализацию и обладает высокой производительностью, однако аппаратная виртуализация еще не поддерживается большинством процессоров.

Возникает задача создания виртуальной машины, которая могла бы работать без аппаратной виртуализации и изменения кода гостевой ОС, и при этом имела достаточно высокую производительность. Предлагается использовать технологию бинарной динамической трансляции, которая удовлетворяет всем указанным выше требованиям. Бинарная трансляция включает в себя несколько этапов, одним из них является декодирование - распознавание и трансляция инструкций исходного кода.

