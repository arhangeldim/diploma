%Virtualization Technics
\section{Техники виртуализации}
Когда архитектура не удовлетворяет формальным требованиям, как, например, x86 и ARM, могут быть использованы другие техники виртуализации:

\begin{enumerate}
    \item эмуляция;
    \item бинарная транcляция;
    \item паравиртуализация;
    \item аппаратная виртуализация.
\end{enumerate}

\textbf{Эмуляция}

Самый простой способ виртуализации - все инструкции гостевого кода передаются  на исполнение в монитор виртуальной машины. Никакие инструкции не исполняются нативно. Такой подход дает высокую надежность, так как монитор полностью контролирует гостевой код, и универсальность - можно исполнять гостевой код на любой платформе. Однако это очень медленно и редко применяется на практике. Примером такого решения является Qemu.

\textbf{Бинарная трансляция}

При таком подходе часть инструкций исполняется нативно, а те sensitive инструкции, которые не являются priveleged, заменяются на priveleged инструкции или эмулируются так, чтобы их можно было выполнить нативно. Обычно результат трансляции сохраняется в кэш-блоках и используется в дальнейшем\cite{bib:vmware_understanding}. Гостевая операционная система не знает, что исполняется под монитором, и не требует модификаций. Бинарная трансляция обеспечивает большую скорость по сравнению с эмуляцией, однако это сложно в реализации. Также возникают накладные расходы на создание кода (трансляцию).

\textbf{Паравиртуализация}

Техника виртуализации, при которой ядро гостевой операционной системы модифицируется, невиртуализуемые инструкции заменяются на специальные вызовы гипервизора - hypercalls. Гипервизор предоставляет интерфейс для работы с памятью, прерываниями, для управления таймерами. Паравиртуализация обеспечивает почти нативную скорость работы. Метод подходит только для систем с открытым исходным кодом, который позволено изменять. Проект Xen\cite{bib:xen_art} - пример использования паравиртуализации  с использованием модифицированного ядра Linux (виртуализована работа с процессором, памятью, вводом/выводом).

\textbf{Аппаратная виртуализация}

Производители аппаратного обеспечения также стараются внедрить в свои продукты поддержку виртуализации. Intel и AMD имеют технологии для аппаратной виртуализации x86 (Intel VT, AMD-V соответственно), ARM планирует поддержку технологии VE(Virtualization Extension) начиная с Cortex A15. Суть всех этих технологий заключается во введении нового режима процессора (hypervisor mode), в котором исполняется монитор виртуальных машин. Новый режим позволяет гостевой системе исполнять свой код (priveleged instructions) нативно, для этого существуют специальные вызовы vmentry/vmexit\cite{bib:vmware_technique}. Помимо этого аппаратные расширения позволяют гостевой системе нативно работать с памятью и прерываниями, например, ARM VE аппаратно поддерживает:

\begin{itemize}
    \item вложенное страничное преобразование;
    \item виртуальный контроллер прерываний и виртуальный таймер;
    \item копии некоторых регистров процессора и сопроцессора.
\end{itemize}
