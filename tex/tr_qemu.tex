\section{Qemu}
Qemu - приложение для эмуляции аппаратного обеспечения различных платформ (x86, PowerPC, ARM, MIPS, SPARC) с открытым исходным кодом. Qemu может быть запущен на Linux, Windows и некоторых UNIX платформах. Используя модуль KVM можно достичь производительности, близкой к нативной.

Qemu поддерживает полную эмуляцию системы, включая процессор, память и различную периферию. Это позволяет запустить несколько различных операционных систем на хостовой системе без перезагрузки. Qemu не требует установки специальных драйверов.

Основные особенности Qemu:
\begin{itemize}
    \item Полная эмуляция;
    \item Динамическая бинарная трансляция;
    \item Поддержка self-modifying кода;
    \item Поддержка исключений;
    \item Поддержка floating point;
    \item Опционально kvm;
\end{itemize}



Qemu состоит из нескольких частей:
\begin{itemize}
    \item CPU emulator (x86, PowerPC, ARM, Sparc);
    \item Device emulator (VGA display, UART, PS/2 mouse, keyboard, network card,..);
    \item Block devices, character devices;
    \item Debugger
\end{itemize}

Внутри Qemu используется динамический транслятор, который позволяет переводить инструкции гостевого процессора в инструкции хостового. Qemu преобразует целевые инструкции в промежуточное представление (Intermediate Language - IL), причем обрабатывает инструкции не по одиночке, а блоками (Translation blocks - TB). На этой стадии применяется некоторая оптимизация кода, включая поиск неисполняемого кода и вычисление констант. Результат трансляции сохраняется в кэш-блоках, и может быть переиспользован в дальнейшем. Такой подход эффективен и обеспечивает переносимость.

Декодирование инструкций в Qemu.

Декодер использует библиотеку opcodes (libopcodes.so) для распознавания инструкций, которая охватывает все существующие наборы инструкций ARM. Для каждой инструкции задается имя, опкод, маска, описание аргументов и обработчик. Все инструкции сохраняются в таблице, сочетание опкода и маски уникально для каждой инструкции. Алгоритм декодирования следующий:

\begin{itemize}
    \item Пройти по списку инструкций и по маске найти соответствующую запись в таблице;
    \item Разобрать аргументы инструкции;
    \item Выполнить трансляцию инструкции;
\end{itemize}

Такой подход позволяет легко добавлять новые инструкции, таблица проста в разработке, но алгоритм поиска имеет сложность O(n), так как в худшем случае придется просмотреть всю таблицу.

Из-за эмуляции Qemu не может добиться высокой производительности, зато обеспечивает поддержку большинства популярных платформ, поэтому его использование оправдано в целях тестирования и отладки системных приложений. Сообщество Android использует Qemu в основе своего Android-эмулятора.






