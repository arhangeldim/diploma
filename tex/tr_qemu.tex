%Technical Review Qemu
\section{Qemu}
Qemu - приложение для эмуляции аппаратного обеспечения различных платформ (x86, PowerPC, ARM, MIPS, SPARC) с открытым исходным кодом. Qemu может быть запущен на Linux, Windows и некоторых UNIX платформах. Используя модуль KVM можно достичь производительности, близкой к нативной. Qemu не требует установки драйвера, поэтому безопасен для хостовой системы.

Qemu поддерживает два режима работы:

\begin{enumerate}
    \item виртуализация уровня платформы эмулирует процессор и основные устройства(Full platform virtualization);
    \item виртуализация уровня приложения позволяет запускать приложения, написанные для одной архитектуры на устройстве другой архитектуры. Требуется, чтобы операционные системы совпадали.
\end{enumerate}

Основные возможности Qemu:
\begin{itemize}
    \item Полная эмуляция;
    \item Динамическая бинарная трансляция;
    \item Поддержка self-modifying кода;
    \item Поддержка исключений;
    \item Поддержка floating point;
    \item Опционально kvm;
\end{itemize}

Qemu состоит из нескольких частей:
\begin{itemize}
    \item CPU emulator (x86, PowerPC, ARM, Sparc);
    \item Device emulator (VGA display, UART, PS/2 mouse, keyboard, network card,..);
    \item Block devices, character devices;
    \item Debugger
\end{itemize}

Внутри Qemu используется динамический транслятор, который позволяет переводить инструкции гостевого процессора в инструкции хостового. Qemu преобразует целевые инструкции в промежуточное представление (Intermediate Language - IL), причем обрабатывает инструкции не по одиночке, а блоками (Translation blocks - TB). На этой стадии применяется некоторая оптимизация кода, включая поиск неисполняемого кода и вычисление констант. Результат трансляции сохраняется в кэш-блоках, и может быть переиспользован в дальнейшем. Такой подход эффективен и обеспечивает переносимость.

{\bf Декодирование инструкций в Qemu}

Преимуществом декодера Qemu является то, что он охватывает все наборы инструкций - ARM, Thumb, Thumb2, VFP, SIMD. Для каждой инструкции задается имя, опкод, маска, описание аргументов и обработчик. Все инструкции сохраняются в таблице, сочетание опкода и маски уникально для каждой инструкции. Алгоритм декодирования следующий:

\begin{enumerate}
    \item Пройти по списку инструкций и по маске найти соответствующую запись в таблице;
    \item Разобрать аргументы инструкции;
    \item Выполнить трансляцию инструкции;
\end{enumerate}

Такой подход позволяет легко добавлять новые инструкции, таблица проста в разработке, но алгоритм поиска имеет сложность O(n), так как в худшем случае придется просмотреть всю таблицу.

Из-за эмуляции Qemu не может добиться высокой производительности, зато обеспечивает поддержку большинства популярных платформ, поэтому его использование оправдано в целях тестирования и отладки системных приложений. Сообщество Android использует Qemu в основе своего Android-эмулятора.






