% Постановка задачи:
% Цель работы
% Конкретная задача, аппарат для решения, новизна работы
% Что и Зачем сделано?

Целью работы является разработка алгоритма декодирования инструкций процессора ARM. Технология динамической бинарной трансляции предполагает исполнение одного набора инструкций на другом, либо изменение некоторых инструкций на другие. В том и в другом случае требуется трансляция машинного кода гостевой системы, а значит эффективность бинарной трансляции зависит от эффективности обработки инструкций исходного кода.

В первом приближении декодирование выглядит как сопоставление опкода с инструкцией из набора и выполнение обработчика этой инструкции. Требуется обеспечить высокую производительность работы декодера (производительность можно оценивать как количество распознанных операций в единицу времени), а значит поиск соответствующей опкоду инструкции должен занимать как можно меньше времени. Особенность архитектуры ARM является использование нескольких наборов инструкций (ISA - Instructions Set Architecture), которые могут исполнятся смешано. Требуется декодировать инструкции из всех доступных наборов. По возможности алгоритм должен обрабатывать наборы инструкций единообразно и иметь возможность расширения новыми наборами инструкций.

В результате исследования должен быть разработан алгоритм декодирования и написан программный модуль, реализующий этот алгоритм. К программному модулю предъявляются следующие требования: единый программный интерфейс для декодирования инструкций из различных наборов, универсальность и платформонезависимость. Это позволит использовать модуль декодирования в различных реализация виртуальной машины без изменений.



