% Using ARM for server
Согласно результатам исследования Amazon, на серверное оборудование приходится около 57\% текущих затрат в ЦОД, основной задачей является повышение экономичности и энергоэффективности серверов\cite{lan_magazine}. Одним из решений является организация серверов на ARM. На данный момент HP и Dell предлагают заказчикам протестировать серверы ARM. На вторую половину 2013 года запланирован выпуск серверов HP на базе Cortex A15, который аппаратно поддерживает виртуализацию. В дальнейшем планируется переход на 64-битную архитектуру ARMv8. 

Недостатком использования ARM является отсутствие широкого спектра программных продуктов (по сравнению с x86), но постепенно крупные компании включают поддержку ARM в свои продукты. Так сейчас ряд дистрибутивов поддерживает ARM и многие ее дополнительные возможности.

Одновременно с этим появляются решения, которые позволят исполнять x86 приложения на ARM. Компания "Эльбрус Технологии" разрабатывает проект по трансляции x86 кода в ARM. На их сайте утверждается, что производительность транслированного кода будет на уровне 80\% от нативного\cite{site_elbrus}. Также они работают над поддержкой расширений SSEx и MMX. 