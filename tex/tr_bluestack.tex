%Technical Review Bluestack
\section{BlueStacks}

BlueStacks AppPlayer позволяет запускать Android приложения на Windows или Mac. На данный момент распространяется бесплатно, исходные коды закрыты. BlueStacks обеспечивает запуск приложений Android в полноэкранном или оконном режимах, синхронизацию с Android устройством, доступ к сервисам через Google-аккаунт, доступ к Google Play. BlueStacks имеет также свой онлайн-сервис BlueStacks Connection Cloud, в облаке можно хранить до 35 приложений. С помощью сервиса можно устанавливать новые приложения, синхронизировать приложения, контакты и другую информацию между различными копиями BlueStacks AppPlayer и реальными Android-устройствами. Поддерживаются многие хостовые устройства ввода - touchscreen (а также акселерометр для tablet-устройств под управлением Windows8), touchpad, клавиатура, мышь.

\textbf{Технология}

BlueStacks использует внутри себя Android 2.3.4 (API 10), ядро Linux 2.6.38-android-x86, но в течение 2013 года планируется переход на Android 4.2 (API 17). Начиная с бета-версии (текущая) BlueStacks поддерживает исполнение native ARM кода, используя бинарную трансляцию. Библиотеки libhoudini.so от Intel для бинарной трансляции ARM/x86 в сборке не найдено.

Предположительно внутри BlueStacks использует переписанное ядро Linux, которое перенаправляет свои вызовы гипервизору. Гипервизор устанавливается в хостовую систему как драйвер (HD-Hypervisor-amd64.sys/HD-Hypervisor-x86.sys). Таким образом используется механизм паравиртуализации. Аппаратная виртуализация не используется (При отключении в BIOS аппаратной виртуализации результаты замеров не изменяются). Из-за использования паравиртуализации сложно поддерживать новые версии операционной системы, поэтому выпуск BlueStacks с Android 4.2 запланирован на вторую половину 2013 года.

В памяти хостовой системы висит несколько процессов, связанных с Bluestack. HD-Adb - сервис отвечает за adb интерфейс, HD-Frontend.exe - интерфейс, HD-Agent.exe - основной фоновый процесс. HD-LogRotator.exe - логирование. HD-BlockDevice.exe - работа с файловой системой Android. Рабочая директория приложения в Windows C:/ProgramData/Bluestacks в ней хранятся логи (для всех сервисов и гипервизора), образ операционной системы, файловая система (initrd.img, kernel.elf, Root.fs, Prebundled.fs).

\textbf{Производительность}

Замеры проводились на PC Windows Vista, AMD Athlon 64 X2 Dual Processor 4800+, 6Gb RAM, NVIDIA GeForce6150 и на Mac 10.6.6 SnowLeopard, Quad-Corei7, 2,2 GHz, 4Gb RAM, AMD Radeon HD 6490M, 1Gb. Сравнение проводилось с Parallels Desktop (на PC 6.0, на Mac 9.0).

Тест Pidigits - вычисляет значение числа ${\pi}$ (первые n символов), тест написан на Java.
Бенчмарк LinpackPro - замеряет скорость операций с плавающей точкой (FLOPS).

Результаты теста на PC Таблица \ref{tab:pc_pidigits}.
Результаты теста на Mac Таблица \ref{tab:mac_pidigits}. Тест в Bluestacks(Mac) работает нестабильно, время от времени приложение завершает работу при большом количестве вычисляемых символов. Возможно это проблемы с выделением памяти(поддержка Mac была добавлена недавно).

По сравнению с QEMU оба решения работают в десятки раз быстрее.

Бенчмарк на Mac дает результат 32.2 MFLOPS(Bluestacks) / 101 MFLOPS(PD 9).

\begin{table}[h]
\caption{\label{tab:pc_pidigits} Pidigits. PC: Bluestacks PC and PD 6.0}
\begin{center}
\begin{tabular} {|c|c|c|}
\hline
Количество символов $\pi$ & Bluestacks, $10^3$ms & PD, $10^3$ms \\
\hline
100 & 0.08 & 0.2 \\
\hline
1000 & 1.2 & 3.0 \\
\hline
10000 & 53 & 94 \\
\hline
\end{tabular}
\end{center}
\end{table}


\begin{table}[h]
\caption{\label{tab:mac_pidigits} Pidigits. Mac: Bluestacks PC and PD 9.0}
\begin{center}
\begin{tabular} {|c|c|c|}
\hline
Количество символов $\pi$ & Bluestacks, $10^3$ms & PD, $10^3$ms \\
\hline
100 & 0.07 & 0.05 \\
\hline
1000 & 1.2 & 0.9 \\
\hline
10000 & 46 & 27 \\
\hline
\end{tabular}
\end{center}
\end{table}

\textbf{Развитие проекта}

BlueStacks в течение 2012 года заключили несколько коммерческих сделок с крупными OEM компаниями - Lenovo, ASUS и с производителями hardware - AMD, Intel. Lenovo и ASUS будут предустанавливать BlueStacks AppPlayer на свои новые ноутбуки, начиная с 2013 года. Сейчас компания получает деньги за рекламу приложений.

AMD использует технологии BlueStacks предлагая пользователям плеер Android-приложений AppZone Player и магазин приложений AppZone. Плеер предустанавливается на ноутбуки, использующие процессоры AMD. Про Intel еще нет достаточной информации. Для корректной работы BlueStacks AppPlayer требуется OpenGL ES 2.0.

В данный момент планируется портирование BlueStacks на Windows RT (в течение 2013 года) и апдейт до версии Android 4.2.




