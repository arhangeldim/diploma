\section{BlueStacks}
BlueStacks AppPlayer позволяет запускать Android приложения на Windows или Mac. На данный момент распространяется бесплатно, исходные коды закрыты. BlueStacks обеспечивает запуск приложений Android в полноэкранном или оконном режимах, синхронизацию с Android устройством, доступ к сервисам через Google-аккаунт, доступ к Google Play. BlueStack имеет также свой онлайн-сервис, на котором создаются учетные записи после регистрации. С помощью сервиса можно устанавливать новые приложения, синхронизировать приложения, контакты и другую информацию между различными копиями BlueStacks AppPlayer и реальными устройствами под управлением Android. Имеется ограничение на количество установленных приложений, это зависит от версии AppPlayer (сейчас это 17 приложений для BlueStacks AppPlayer for Mac).

На официальном сайте говорится, что BlueStacks использует гипервизор и это позволяет добиться большей производительности по сравнению с эмуляцией. Сейчас BlueStacks использует внутри себя Android 2.3.4 (API 10), но в течение 2013 года планируется переход на Android 4.2 (API 17). Начиная с бета-версии (текущая) BlueStacks поддерживает исполнение native ARM кода, используя бинарную трансляцию. BlueStacks работает с различными комбинациями операционных систем и аппаратных архитектур.

На данный момент поддерживаются:
\begin{itemize}
    \item Android on Windows (x86);
    \item Android on Windows (ARM since Windows 8 release);
    \item Android on Chrome OS (x86);
    \item Windows on Android (x86);
\end{itemize}

В среднем скорость работы выше, чем у эмулятора Android. BlueStacks поддерживает adb интерфейс и может использоваться для отладки приложений.

BlueStacks в течение 2012 года заключили несколько коммерческих сделок с крупными компьютерными компаниями - Lenovo, ASUS, AMD, Intel. Lenovo и ASUS будут предустанавливать BlueStacks AppPlayer на свои новые ноутбуки.

Партнерство BlueStacks и AMD/Intel в основном нацелено на оптимизацию Android приложений для AMD/Intel-based Windows устройств. BlueStacks использует внутреннюю закрытую технологию Layercake, которая позволяет улучшить производительность Android приложений, игр в Windows (в будущем, видимо, и в OS X). Android приложения обычно используют возможности графического ускорителя (SGX), установленного на плате (ARM Mali, PowerVR, Nvidia Tegra), Layercake позволяет использовать при виртуализации графические ускорители AMD (GPU/APU технологии)на x86. Про Intel еще нет достаточной информации. Для корректной работы BlueStacks AppPlayer требуется OpenGL ES 2.0.

В данный момент планируется портирование приложения на Windows RT (в течение 2013 года).




