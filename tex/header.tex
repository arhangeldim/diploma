\documentclass[a4paper,12pt]{extreport}
\usepackage[T2A]{fontenc}
\usepackage[utf8]{inputenc}
\usepackage[english,russian]{babel}
\usepackage[left=2cm,right=1cm,top=2cm,bottom=2cm]{geometry}
\usepackage{indentfirst}
\usepackage{amssymb, soul}
\usepackage{amsmath}

\usepackage{pscyr}

% гиперссылки в документе
%\usepackage[unicode]{hyperref}

%стиль библиографии
\bibliographystyle{unsrt}

\makeatletter
\renewcommand{\@biblabel}[1]{#1.} % Заменяем библиографию с квадратных скобок на точку:
\makeatother

%межстрочный интервал
\renewcommand{\baselinestretch}{1.65}

%значки больше-меньше
\renewcommand{\ge}{\geqslant}
\renewcommand{\le}{\leqslant}

%изменение оформления глав
\makeatletter
\renewcommand{\@makechapterhead}[1]{
\vspace{10pt}
{\parindent=0pt
\raggedright \normalfont \Large \bfseries
\thechapter. \hspace{5 pt}
\normalfont \Large \bfseries #1\par
\nopagebreak
\vspace{10pt}
}}

\renewcommand{\@makeschapterhead}[1]{
\vspace{10pt}
{\parindent=0pt
\raggedright \normalfont \Large \bfseries
\normalfont \Large \bfseries #1\par
\nopagebreak
\vspace{10pt}
}}

\renewcommand{\@listI}{%
\leftmargin=25pt
\rightmargin=0pt
\labelsep=5pt
\labelwidth=20pt
\itemindent=0pt
\listparindent=0pt
\topsep=8pt plus 2pt minus 4pt
\partopsep=2pt plus 1pt minus 1pt
\parsep=0pt plus 1pt
\itemsep=\parsep}

\renewcommand{\chapter}{
\global\@topnum=0
\@afterindenttrue
\secdef\@chapter\@schapter}

\makeatother

\begin{document}
\renewcommand{\bibname}{Список источников}
\renewcommand{\contentsname}{\center{СОДЕРЖАНИЕ}}

\tableofcontents

\newpage
\chapter{Введение}
    % Введение :
%   Область исследования
%   Проблема
%   Актуальность
%   Конкретная проблема, решаемая в работе

Данная работа является исследованием в области виртуализации архитектуры ARM. Термин виртуализация означает различные методы представления вычислительных ресурсов(например, аппаратная платформа, операционная система, хранилище данных), которые дают преимущества перед их оригинальной конфигурацией. Виртуализация - это общая концепция для многих аспектов вычислений, более подробно типы виртуализации будут рассмотрены ниже.

Архитектура ARM - семейство RISC процессоров, разрабатываемых британской компанией ARM Limited. Большинство современных мобильных устройств - смартфонов и планшетов - используют процессоры ARM и этот показатель постоянно растет. Например, только за первый квартал 2013 года было выпущено 2,6 млрд. процессоров\cite{bib:arm_web}.

Процессоры ARM отличаются низкой стоимостью, малым энергопотреблением и тепловыделением, что позволят использовать их в мобильных устройствах и встраиваемых системах. ARM используется в качестве процессора Apple iPad, Apple iPhone, Microsoft Surface, Samsung Galaxy. Также ARM-чипы используются в SOC системах Raspberry Pi, PandaBoard и других.

В отношении мобильных устройств возникает несколько проблем. Сейчас развивается технология BYOD(Bring your own device) - использование одного устройства для личных и корпоративных целей, при этом встает проблема изоляции двух окружений, с одной стороны безопасность данных компании, а с другой стороны приватность данных владельца. Обычно у одного владельца есть несколько мобильных устройств, например, телефон и планшет, тогда возникает задача миграции данных между устройствами, также при замене устройства было бы удобно переносить все данные и настройки со старого устройства на новое. Еще одна проблема - непереносимость приложений между мобильными операционными системами, виртуализация могла бы решить эту проблему, создав окружение, в котором можно запускать приложения, написанные для другой ОС.

Также, в 2012 году были представлены сервера на базе ARM, это означает возможность применения виртуализации для более эффективного управления ресурсами.

Сейчас существует несколько решений для виртуализации архитектуры ARM. Qemu полностью виртуализует платформу, Xen ARM и KVM ARM используют паравиртуализацию и аппаратную виртуализацию. Подход Qemu универсален, но требует полной эмуляции платформы, а значит обладает невысокой производительностью, Xen ARM имеет высокую скорость работы, но при этом требуется вносить изменения в гостевую ОС, KVM ARM использует аппаратную виртуализацию и обладает высокой производительностью, однако аппаратная виртуализация еще не поддерживается большинством процессоров.

Возникает задача создания виртуальной машины, которая могла бы работать без аппаратной виртуализации и изменения кода гостевой ОС, и при этом имела достаточно высокую производительность. Предлагается использовать технологию бинарной динамической трансляции, которая удовлетворяет всем указанным выше требованиям. Бинарная трансляция включает в себя несколько этапов, одним из них является декодирование - распознавание и трансляция инструкций исходного кода.



\newpage
\chapter{Постановка задачи}
    % Постановка задачи:
% Цель работы
% Конкретная задача, аппарат для решения, новизна работы
% Что и Зачем сделано?

Целью работы является разработка алгоритма декодирования инструкций процессора ARM. Технология динамической бинарной трансляции предполагает исполнение одного набора инструкций на другом, либо изменение некоторых инструкций на другие. В том и в другом случае требуется трансляция машинного кода гостевой системы, а значит эффективность бинарной трансляции зависит от эффективности обработки инструкций исходного кода.

В первом приближении декодирование выглядит как сопоставление опкода с инструкцией из набора и выполнение обработчика этой инструкции. Требуется обеспечить высокую производительность работы декодера (производительность можно оценивать как количество распознанных операций в единицу времени), а значит поиск соответствующей опкоду инструкции должен занимать как можно меньше времени. Особенность архитектуры ARM является использование нескольких наборов инструкций (ISA - Instructions Set Architecture), которые могут исполнятся смешано. Требуется декодировать инструкции из всех доступных наборов. По возможности алгоритм должен обрабатывать наборы инструкций единообразно и иметь возможность расширения новыми наборами инструкций.

В результате исследования должен быть разработан алгоритм декодирования и написан программный модуль, реализующий этот алгоритм. К программному модулю предъявляются следующие требования: единый программный интерфейс для декодирования инструкций из различных наборов, универсальность и платформонезависимость. Это позволит использовать модуль декодирования в различных реализация виртуальной машины без изменений.





\newpage
\chapter{Литературный обзор}
    \section{Формальные требования к виртуализации}
        %Popek and Goldberg formal virtualization requirements

Первое описание формальных требований, которым должна удовлетворять архитектура для поддержки ее виртуализации, было опубликовано в 1974 в статье Gerald J. Popek and Robert P. Goldberg "Formal Requirements for Virtualizable Third Generation Architectures" \cite{bib:popek_goldberg}.

Монитор виртуальных машин (монитор, гипервизор) - программа, позволяющая запустить несколько операционных систем на одном хостовом компьютере. Монитор должен обладать следующими свойствами:

\begin{itemize}
    \item идентичность, эквивалентность - программа, запущенная под управлением монитора, должна вести себя идентично программе, запущенной на реальном оборудовании;
    \item управление ресурсами - монитор полностью контролирует свои ресурсы;
    \item производительность - значительная часть гостевых инструкций должна выполняться без вмешательства монитора;
\end{itemize}

Также описываются требования к набору инструкций, которым должна удовлетворять архитектура. Инструкции делятся на три группы:

\begin{enumerate}
    \item privileged - выполняются в привилегированном режиме;
    \item control sensitive - могут изменить режим или состояние процессора;
    \item behavior sensitive - поведение инструкции зависит от состояния процессора;
\end{enumerate}

\textbf{Теорема: } если sensitive инструкции являются подмножеством priveleged инструкций, то архитектура виртуализуема.

Это означает, что при обработке sensitive инструкций управление может быть передано монитору. Непривилегированные инструкции могут исполняться нативно. Такая техника называется trap-and-emulate virtualization или классическая виртуализация. Архитектура ARM не является виртуализуемой, так как часть sensitive инструкций не является priveleged.

Для более формального определения инструкции, нужно ввести понятие модели процессора. Пусть процессор может работать в одном из двух режимов - supervisor и user. В режиме supervisor ему доступен весь набор инструкций, в режиме user нет. Процессор может также находиться в одном из конечного множества состояний. Каждое состояние S определяется четырьмя компонентами\cite{bib:popek_goldberg}: состояние памяти \textit{E}, режим процессора \textit{M}, указатель инструкций \textit{P}, регистр смещения адреса \textit{R}.

\begin{equation}\label{eq:machine_state}
    S=\left<E, M, P, R\right>
\end{equation}

$E[i]$ - означает $i$-й элемент памяти. Причем, для $E$ размера $q$: $E=E' \Leftrightarrow E[i] = E'[i]$, для любого $0 \le i  \le q$. $R=(l, b)$ состоит из двух частей, $l$ - база, $b$ - лимит. Обозначим конечное множество состояний как $C$, тогда инструкция может быть определена как функция из $C$ в $C$, т.е.  $i:C\to C$. Например, $i(S_1) = S_2,\ i(E_1, M_1, P_1, R_1) = (E_2, M_2, P_2, R_2)$. Для определения привилегированных инструкций потребуется понятие исключения (\textit{trap}).

\textbf{Определение}: инструкция вызывает исключение (\textit{trap}), если $i(S_1) = S_2,\ i(E_1, M_1, P_1, R_1) = (E_2, M_2, P_2, R_2)$, где

\begin{equation*}
    \begin{aligned}
        &E_2[j] = E_1[j],\ 0 \le j \le q,\\
        &E_2[0] = (M_1, P_1, R_1),\\
        &(M_2, P_2, R_2) = E_1[1].
    \end{aligned}
\end{equation*}

Таким образом, исключение сохраняет текущий контекст и передает управление специальной процедуре, изменяя внутренне состояние процессора $(M, P, R)$.

\textbf{Определение}: инструкция называется привилегированной (privileged) тогда и только тогда, когда для любой пары состояний $S_1=\left<e, s, p, r\right>$ и $S_2=\left<e, u, p, r\right>$ $i(S_2)$ вызывает исключение(\textit{trap}), а $i(S_1)$ - нет. Состояния $S_1, S_2$ различаются лишь режимом процессора: $S_1 - supervisor, S_2 - user$.

Другая важная группа инструкций - sensitive инструкции.

\textbf{Определение}: инструкция $i$ является \textit{control sensitive}, если существует состояние $S_1=\left<e_1, m_1, p_1, r_1\right>$, и $i(S_1) = S_2 = \left<e_2, m_2, p_2, r_2\right>$ такая, что $a)\ r1 \ne r2$, или $b)\ m_1 \ne m_2$, или $a),\ b)$ совместно.

Также определим для любого целого $x$ операцию $\otimes$ : $r' = r \otimes x = (l+x, b)$. $R=(l, b)$ - определяет, к каким адресам памяти можно обращаться из состояния $R$. Таким образом $E|r \otimes x$ обозначает, что доступны адреса с $|l + x|$ по $|l + x + b|$. Теперь можно ввести новый класс инструкций.

\textbf{Определение}: Инструкция $i$ является \textit{behavior sensitive}, если существует целое $x$ и состояния:

\begin{equation*}
    \begin{aligned}
        &S_1=\left<e | r, m_1, p, r\right>, \\
        &S_2=\left<e | r, m_2, p, r\right>,
    \end{aligned}
\end{equation*}

где

\begin{equation*}
    \begin{aligned}
        &i(S_1) = \left<e_1 | r, M_1, P_1, r\right>, \\
        &i(S_2) = \left<e_2 | r \otimes x, m_2, p_2, r \otimes x\right>,
    \end{aligned}
\end{equation*}

удовлетворяют условиям

\begin{equation*}
    \begin{aligned}
        &e_1 | r \ne e_2 | r \otimes x, или \\
        &p_1 \ne p_2, или \\
        &оба совместно
    \end{aligned}
\end{equation*}


    
    \newpage    
    \section{Техники виртуализации}    
        %Virtualization Technics

Когда архитектура не удовлетворяет формальным требованиям, как, например, x86 и ARM, могут быть использованы другие техники виртуализации:

\begin{enumerate}
    \item эмуляция;
    \item бинарная транcляция;
    \item паравиртуализация;
    \item аппаратная виртуализация.
\end{enumerate}

\textbf{Эмуляция}

Самый простой способ виртуализации - все инструкции гостевого кода передаются  на исполнение в монитор виртуальной машины. Никакие инструкции не исполняются нативно. Такой подход дает высокую надежность, так как монитор полностью контролирует гостевой код, и универсальность - можно исполнять гостевой код на любой платформе. Однако это очень медленно и редко применяется на практике. Примером такого решения является Qemu.

\textbf{Бинарная трансляция}

При таком подходе часть инструкций исполняется нативно, а те sensitive инструкции, которые не являются priveleged, заменяются на priveleged инструкции или эмулируются так, чтобы их можно было выполнить нативно. Обычно результат трансляции сохраняется в кэш-блоках и используется в дальнейшем\cite{bib:vmware_understanding}. Гостевая операционная система не знает, что исполняется под монитором, и не требует модификаций. Бинарная трансляция обеспечивает большую скорость по сравнению с эмуляцией.

\textbf{Паравиртуализация}

Техника виртуализации, при которой ядро гостевой операционной системы модифицируется, невиртуализуемые инструкции заменяются на специальные вызовы гипервизора - hypercalls. Гипервизор предоставляет интерфейс для работы с памятью, прерываниями, для управления таймерами. Паравиртуализация обеспечивает почти нативную скорость работы. Метод подходит только для систем с открытым исходным кодом, который позволено изменять. Проект Xen\cite{bib:xen_art} - пример использования паравиртуализации  с использованием модифицированного ядра Linux (виртуализована работа с процессором, памятью, вводом/выводом).

\textbf{Аппаратная виртуализация}

Производители аппаратного обеспечения также стараются внедрить в свои продукты поддержку виртуализации. Intel и AMD имеют технологии для аппаратной виртуализации x86 (Intel VT, AMD-V соответственно), ARM планирует поддержку технологии VE(Virtualization Extension) начиная с Cortex A15. Суть всех этих технологий заключается во введении нового режима процессора (hypervisor mode), в котором исполняется монитор виртуальных машин. Новый режим позволяет гостевой системе исполнять свой код (priveleged instructions) нативно, для этого существуют специальные вызовы vmentry/vmexit\cite{bib:vmware_technique}. Помимо этого аппаратные расширения позволяют гостевой системе нативно работать с памятью и прерываниями, например, ARM VE аппаратно поддерживает:

\begin{itemize}
    \item вложенное страничное преобразование;
    \item виртуальный контроллер прерываний и виртуальный таймер;
    \item копии некоторых регистров процессора и сопроцессора.
\end{itemize}


    \newpage
    \section{Обзор архитектуры ARM}
        %ARM Architecture Review

\subsection{Особенности ARM}
Архитектура ARM это RISC архитектура и обладает следующими особенностями, присущими RISC\cite{bib:arm_arch_manual}:

\begin{itemize}
    \item load/store архитектура (инструкции обработки данных работают с регистрами, а не напрямую с памятью);
    \item простые режимы адресации (все load/store адреса однозначно определяются аргументами инструкции);
    \item фиксированная длина команд для упрощения декодирования;
    \item исполнение команды за один цикл.
\end{itemize}

Также ARM обеспечивает:

\begin{itemize}
    \item инструкции, которые комбинируют логические/арифметические операции и сдвиговые;
    \item режимы адресации с инкрементом;
    \item блочные load/store операции;
    \item условное выполнение для многих инструкций.
\end{itemize}

За время существования ARM было разработано семь версий архитектуры, сейчас поддерживаются архитектуры выше ARMv3. Начиная с ARMv7 введено дополнительное разделение на профили: Application(ARMv7-A), RealTime(ARMv7-R) и Microcontroller(ARMv7-M).

Процессор может находиться в одном из следующих режимов:
\begin{itemize}
    \item User - обычный режим выполнения программ;
    \item FIQ - обработка высокоприоритетного прерывания;
    \item IRQ - обработка низкоприоритетного прерывания;
    \item System - привилегированный режим;
    \item Abort - ошибка доступа к памяти или данным;
    \item Supervisor - Software Interrupt Instruction (SWI);
    \item Undefined - обработка неизвестной инструкции.
\end{itemize}

В каждом режиме доступен собственный стек и набор регистров. Некоторые инструкции могут исполняться только в привилегированном режиме, некоторые инструкции могут по разному исполняться в разных режимах. Суммарно во всех режимах доступно 37 регистров (32 бита), из них 31 регистр общего назначения и 6 - статусные. В режиме User доступно 13 регистров общего назначения R0-R12 и 3 специальных регистра SP, LR, PC. Для каждого режима имеются отдельные регистры SP, LR, PC.

\begin{itemize}
    \item SP (Stack Pointer) - хранит указатель на текущий стек;
    \item LR (Link Register) - хранит адрес возврата, также можно использовать в качестве регистра общего назначения (R14);
    \item PC (Program Counter) - в режиме исполнения ARM указывает на текущую инструкцию~+8, в режиме Thumb соответственно~+4, это связано с устройством конвейера инструкций (он трехстадийный - fetch/decode/execute).
\end{itemize}

Регистр CPSR(Current Program Status Register) содержит флаги, описывающие текущее состояние процессора: информация о последних арифметических операциях (переполнение, знак,..), управление прерываниями, режим работы процессора, Thumb/ARM режим. При смене режима процессора текущее содержимое CPSR сохраняется в SPSR, соответствующий текущему режиму. При возвращении к исходному режиму состояние CPSR восстанавливается.

В ARM доступно условное исполнение команд. Это означает, что инструкция будет иметь эффект только тогда, когда состояние процессора (флаги CPSR) удовлетворяет условию, закодированному в поле {\it cond} инструкции. Если флаги не соответствуют условию, инструкция исполняется как NOP, то есть не имеет эффекта и управление передается дальше.

\subsection{Наборы инструкций}

Архитектура ARMv7 поддерживает два основных набора команд - ARM и Thumb. Оба набора предоставляет похожие возможности - арифметические и логические операции, работа с памятью, переходы, работа с сопроцессором.

\textbf{ARM}

Инструкции имеют фиксированный размер 32 бита, должны быть выровнены в памяти на 4 байта, могут выполняться условно (биты <31:28> - поле \textit{cond}).

\textbf{Thumb}

Для увеличения плотности кода был создан усеченный набор инструкций, они имею фиксированный размер 16 бит, однако обладают меньшей функциональностью. Например, только команды ветвления могут исполняться условно, инструкции имеют доступ только к регистрам R0-R7. Инструкции должны быть выровнены в памяти на 2 байта.

\textbf{Thumb2}

Технология расширяет набор Thumb дополнительными 32-битными командами, добавляя функциональность набора ARM. Многие операции требующие нескольких Thumb команд, могут быть выполнены эффективнее с использованием Thumb2. Инструкции имеют доступ ко всем регистрам общего назначения.

Процессор может исполнять ARM и Thumb инструкции совместно (находясь в ARM и Thumb режимах соответственно). Процессор может изменить режим выполнив команды BX, BLX, LDR/LDM(если изменяют PC). Начиная с ARMv7 можно изменить состояние с помощью инструкций обработки данных (ADC, ADD, AND, MOV,..), если они изменяют PC.

\subsection{Сопроцессоры}

Архитектура позволяет расширять функциональность, использую сопроцессоры. Может быть адресовано 16 сопроцессоров CP0-CP15, обратиться к ним можно с помощью инструкций MRC, MCR, LDC, STC, CDP. Сопроцессор управления системой (CP15) отвечает за кэш-память и виртуальную память.
Advanced SIMD - набор SIMD (Single Instruction Multiple Data) инструкций, который обеспечивает ускорение для мультимедиа приложений и обработки сигналов. Поддерживает работу с 8-, 16-, 32-, 64-битными элементами целого типа, одинарной точности и с плавающей запятой. VFP(Vector Floating Point) - операции над числами с плавающей запятой в соответствии с ANSI/IEEE-754.






    \newpage
    \section{Обзор существующих решений}
        \section{Qemu}
Qemu - приложение для эмуляции аппаратного обеспечения различных платформ (x86, PowerPC, ARM, MIPS, SPARC) с открытым исходным кодом. Qemu может быть запущен на Linux, Windows и некоторых UNIX платформах. Используя модуль KVM можно достичь производительности, близкой к нативной.

Qemu состоит из нескольких частей:
\begin{itemize}
    \item CPU emulator (x86, PowerPC, ARM, Sparc);
    \item Device emulator (VGA display, UART, PS/2 mouse, keyboard, network card,..);
    \item Block devices, character devices;
    \item Debugger
\end{itemize} 

Внутри Qemu используется динамический транслятор, который позволяет переводить инструкции гостевого процессора в инструкции хостового. Qemu преобразует целевые инструкции в промежуточное представление (Intermediate Language - IL), причем обрабатывает инструкции не по одиночке, а блоками (Translation blocks - TB). На этой стадии применяется некоторая оптимизация кода, включая поиск неисполняемого кода и вычисление констант. Результат трансляции хранится в кэш-блоках, и может быть переиспользован в дальнейшем. Такой подход эффективен и обеспечивает переносимость.






        %Technical Review Bluestack
\section{BlueStacks}

BlueStacks AppPlayer позволяет запускать Android приложения на Windows или Mac. На данный момент распространяется бесплатно, исходные коды закрыты. BlueStacks обеспечивает запуск приложений Android в полноэкранном или оконном режимах, синхронизацию с Android устройством, доступ к сервисам через Google-аккаунт, доступ к Google Play. BlueStacks имеет также свой онлайн-сервис BlueStacks Connection Cloud, в облаке можно хранить до 35 приложений. С помощью сервиса можно устанавливать новые приложения, синхронизировать приложения, контакты и другую информацию между различными копиями BlueStacks AppPlayer и реальными Android-устройствами. Поддерживаются многие хостовые устройства ввода - touchscreen (а также акселерометр для tablet-устройств под управлением Windows8), touchpad, клавиатура, мышь.

\textbf{Технология}

BlueStacks использует внутри себя Android 2.3.4 (API 10), ядро Linux 2.6.38-android-x86, но в течение 2013 года планируется переход на Android 4.2 (API 17). Начиная с бета-версии (текущая) BlueStacks поддерживает исполнение native ARM кода, используя бинарную трансляцию. Библиотеки libhoudini.so от Intel для бинарной трансляции ARM/x86 в сборке не найдено.

Предположительно внутри BlueStacks использует переписанное ядро Linux, которое перенаправляет свои вызовы гипервизору. Гипервизор устанавливается в хостовую систему как драйвер (HD-Hypervisor-amd64.sys/HD-Hypervisor-x86.sys). Таким образом используется механизм паравиртуализации. Аппаратная виртуализация не используется (При отключении в BIOS аппаратной виртуализации результаты замеров не изменяются). Из-за использования паравиртуализации сложно поддерживать новые версии операционной системы, поэтому выпуск BlueStacks с Android 4.2 запланирован на вторую половину 2013 года.

В памяти хостовой системы висит несколько процессов, связанных с Bluestack. HD-Adb - сервис отвечает за adb интерфейс, HD-Frontend.exe - интерфейс, HD-Agent.exe - основной фоновый процесс. HD-LogRotator.exe - логирование. HD-BlockDevice.exe - работа с файловой системой Android. Рабочая директория приложения в Windows C:/ProgramData/Bluestacks в ней хранятся логи (для всех сервисов и гипервизора), образ операционной системы, файловая система (initrd.img, kernel.elf, Root.fs, Prebundled.fs).

\textbf{Производительность}

Замеры проводились на PC Windows Vista, AMD Athlon 64 X2 Dual Processor 4800+, 6Gb RAM, NVIDIA GeForce6150 и на Mac 10.6.6 SnowLeopard, Quad-Corei7, 2,2 GHz, 4Gb RAM, AMD Radeon HD 6490M, 1Gb. Сравнение проводилось с Parallels Desktop (на PC 6.0, на Mac 9.0).

Тест Pidigits - вычисляет значение числа ${\pi}$ (первые n символов), тест написан на Java.
Бенчмарк LinpackPro - замеряет скорость операций с плавающей точкой (FLOPS).

Результаты теста на PC Таблица \ref{tab:pc_pidigits}.
Результаты теста на Mac Таблица \ref{tab:mac_pidigits}. Тест в Bluestacks(Mac) работает нестабильно, время от времени приложение завершает работу при большом количестве вычисляемых символов. Возможно это проблемы с выделением памяти(поддержка Mac была добавлена недавно).

По сравнению с QEMU оба решения работают в десятки раз быстрее.

Бенчмарк на Mac дает результат 32.2 MFLOPS(Bluestacks) / 101 MFLOPS(PD 9).

\begin{table}[h]
\caption{\label{tab:pc_pidigits} Pidigits. PC: Bluestacks PC and PD 6.0}
\begin{center}
\begin{tabular} {|c|c|c|}
\hline
Количество символов $\pi$ & Bluestacks, $10^3$ms & PD, $10^3$ms \\
\hline
100 & 0.08 & 0.2 \\
\hline
1000 & 1.2 & 3.0 \\
\hline
10000 & 53 & 94 \\
\hline
\end{tabular}
\end{center}
\end{table}


\begin{table}[h]
\caption{\label{tab:mac_pidigits} Pidigits. Mac: Bluestacks PC and PD 9.0}
\begin{center}
\begin{tabular} {|c|c|c|}
\hline
Количество символов $\pi$ & Bluestacks, $10^3$ms & PD, $10^3$ms \\
\hline
100 & 0.07 & 0.05 \\
\hline
1000 & 1.2 & 0.9 \\
\hline
10000 & 46 & 27 \\
\hline
\end{tabular}
\end{center}
\end{table}

\textbf{Развитие проекта}

BlueStacks в течение 2012 года заключили несколько коммерческих сделок с крупными OEM компаниями - Lenovo, ASUS и с производителями hardware - AMD, Intel. Lenovo и ASUS будут предустанавливать BlueStacks AppPlayer на свои новые ноутбуки, начиная с 2013 года. Сейчас компания получает деньги за рекламу приложений.

AMD использует технологии BlueStacks предлагая пользователям плеер Android-приложений AppZone Player и магазин приложений AppZone. Плеер предустанавливается на ноутбуки, использующие процессоры AMD. Про Intel еще нет достаточной информации. Для корректной работы BlueStacks AppPlayer требуется OpenGL ES 2.0.

В данный момент планируется портирование BlueStacks на Windows RT (в течение 2013 года) и апдейт до версии Android 4.2.





        % Technical review - kvm/arm
\subsection{KVM/ARM, Xen ARM}
Существует два похожих решения, использующие поддержку аппаратной виртуализации. ARM Virtualization Extension(VE) - расширения для аппаратной виртуализации ARM. VE есть только на процессорах нового поколения (Cortex A15), которые появились в начале 2013 года и еще не используются в коммерческих продуктах. ARM VE добавляет в процессор новый режим гипервизора (hypervisor, hyp). Все новые возможности доступны только из этого режима - отдельное преобразование страниц для гостя, виртуальный контроллер прерываний, копии некоторых регистров процессора.

\textbf{KVM/ARM} 

KVM(Kernel-based Virtual Machine)\cite{bib:kvm} - программное решение, обеспечивающее виртуализацию  в среде Linux. Компоненты ядра, необходимые для KVM, включены в основное ядро начиная с версии 2.6.20.


KVM ARM\cite{bib:kvm_arm} не требует изменять код гостевой ОС, требуется только версия ядра выше 2.6.20, собранная с KVM модулем и с драйверами virtio. Virtio - библиотека для виртуализации ввода/вывода, добавляет интерфейс виртуальных устройств в систему(например, блочные устройства, сетевые устройства). Таким образом, гостевая система виртуализуется аппаратно, только ввод/вывод перенаправляется через virtio в хост, где эмулируется через Qemu.

KVM ARM обладает хорошей производительностью, не требует модификации кода гостя. Может работать только с ОС на ядре Linux и требует поддержки аппаратной виртуализации(процессоры Cortex-A15). KVM хорошо работает в FastModel(программная эмуляция платформы), но на реальном процессоре ведет себя нестабильно.

\textbf{XEN ARM}

Проект Xen ARM\cite{bib:xen_arm} - это вариант Xen Hypervisor для архитектуры ARM. Требует запуска в hypervisor режиме. В качестве гостевой системы для Xen ARM выступает паравиртуализованный Linux - Dom0. Xen планирует внедрять это решение для серверов на базе ARM.






\newpage
\chapter{Методика решения задачи}
    % Методика решения задачи
% Формализация => модель => результат
% Общий алгоритм, ограничения и допущения, технические средства

 

\newpage
\bibliography{arm_biblio}
\end{document}
