%Popek and Goldberg formal virtualization requirements
Первое описание формальных требований, которым должна удовлетворять архитектура для поддержки ее виртуализации, было опубликовано в 1974 в статье {\it Gerald J. Popek and Robert P. Goldberg "Formal Requirements for Virtualizable Third Generation Architectures"}\cite{popek_goldberg}. Монитор виртуальных машин (монитор, гипервизор) - программа, позволяющая запустить несколько операционных систем на одном хостовом компьютере. Монитор должен обладать следующими свойствами:

\begin{itemize}
    \item идентичность, эквивалентность - программа, запущенная под управлением монитора, должна вести себя идентично программе, запущенной на реальном оборудовании;
    \item управление ресурсами - монитор полностью контролирует свои ресурсы;
    \item производительность - значительная часть гостевых инструкций должна выполняться без вмешательства монитора;
\end{itemize}

Также описываются требования к набору инструкций, которым должна удовлетворять физическая машина. Инструкции делятся на три группы:

\begin{enumerate}
    \item privileged - выполняются в привилегированном режиме;
    \item control sensitive - могут изменить режим или состояние процессора;
    \item behavior sensitive - поведение инструкции зависит от состояния процессора;
\end{enumerate}

{\bf Теорема: } если sensitive инструкции являются подмножеством priveleged инструкций, то архитектура виртуализуема.

Это означает, что все при обработке sensitive инструкций управление может быть передано монитору. Непривилегированные инструкции могут исполняться нативно. Такая техника называется trap-and-emulate virtualization или классическая виртуализация. Архитектура ARM не является виртуализуемой, так как часть sensitive инструкций не является priveleged.

