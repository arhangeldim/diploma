%ARM Architecture Review

\subsection{Особенности ARM}
Архитектура ARM это RISC архитектура и обладает следующими особенностями, присущими RISC\cite{bib:arm_arch_manual}:

\begin{itemize}
    \item load/store архитектура (инструкции обработки данных работают с регистрами, а не напрямую с памятью);
    \item простые режимы адресации (все load/store адреса однозначно определяются аргументами инструкции);
    \item фиксированная длина команд для упрощения декодирования;
    \item исполнение команды за один цикл.
\end{itemize}

Также ARM обеспечивает:

\begin{itemize}
    \item инструкции, которые комбинируют логические/арифметические операции и сдвиговые;
    \item режимы адресации с инкрементом;
    \item блочные load/store операции;
    \item условное выполнение для многих инструкций.
\end{itemize}

За время существования ARM было разработано семь версий архитектуры, сейчас поддерживаются архитектуры выше ARMv3. Начиная с ARMv7 введено дополнительное разделение на профили: Application(ARMv7-A), RealTime(ARMv7-R) и Microcontroller(ARMv7-M).

Процессор может находиться в одном из следующих режимов:
\begin{itemize}
    \item User - обычный режим выполнения программ;
    \item FIQ - обработка высокоприоритетного прерывания;
    \item IRQ - обработка низкоприоритетного прерывания;
    \item System - привилегированный режим;
    \item Abort - ошибка доступа к памяти или данным;
    \item Supervisor - Software Interrupt Instruction (SWI);
    \item Undefined - обработка неизвестной инструкции.
\end{itemize}

В каждом режиме доступен собственный стек и набор регистров. Некоторые инструкции могут исполняться только в привилегированном режиме, некоторые инструкции могут по разному исполняться в разных режимах. Суммарно во всех режимах доступно 37 регистров (32 бита), из них 31 регистр общего назначения и 6 - статусные. В режиме User доступно 13 регистров общего назначения R0-R12 и 3 специальных регистра SP, LR, PC. Для каждого режима имеются отдельные регистры SP, LR, PC.

\begin{itemize}
    \item SP (Stack Pointer) - хранит указатель на текущий стек;
    \item LR (Link Register) - хранит адрес возврата, также можно использовать в качестве регистра общего назначения (R14);
    \item PC (Program Counter) - в режиме исполнения ARM указывает на текущую инструкцию~+8, в режиме Thumb соответственно~+4, это связано с устройством конвейера инструкций (он трехстадийный - fetch/decode/execute).
\end{itemize}

Регистр CPSR(Current Program Status Register) содержит флаги, описывающие текущее состояние процессора: информация о последних арифметических операциях (переполнение, знак,..), управление прерываниями, режим работы процессора, Thumb/ARM режим. При смене режима процессора текущее содержимое CPSR сохраняется в SPSR, соответствующий текущему режиму. При возвращении к исходному режиму состояние CPSR восстанавливается.

В ARM доступно условное исполнение команд. Это означает, что инструкция будет иметь эффект только тогда, когда состояние процессора (флаги CPSR) удовлетворяет условию, закодированному в поле {\it cond} инструкции. Если флаги не соответствуют условию, инструкция исполняется как NOP, то есть не имеет эффекта и управление передается дальше.

\subsection{Наборы инструкций}

Архитектура ARMv7 поддерживает два основных набора команд - ARM и Thumb. Оба набора предоставляет похожие возможности - арифметические и логические операции, работа с памятью, переходы, работа с сопроцессором.

\textbf{ARM}

Инструкции имеют фиксированный размер 32 бита, должны быть выровнены в памяти на 4 байта, могут выполняться условно (биты <31:28> - поле \textit{cond}).

\textbf{Thumb}

Для увеличения плотности кода был создан усеченный набор инструкций, они имею фиксированный размер 16 бит, однако обладают меньшей функциональностью. Например, только команды ветвления могут исполняться условно, инструкции имеют доступ только к регистрам R0-R7. Инструкции должны быть выровнены в памяти на 2 байта.

\textbf{Thumb2}

Технология расширяет набор Thumb дополнительными 32-битными командами, добавляя функциональность набора ARM. Многие операции требующие нескольких Thumb команд, могут быть выполнены эффективнее с использованием Thumb2. Инструкции имеют доступ ко всем регистрам общего назначения.

Процессор может исполнять ARM и Thumb инструкции совместно (находясь в ARM и Thumb режимах соответственно). Процессор может изменить режим выполнив команды BX, BLX, LDR/LDM(если изменяют PC). Начиная с ARMv7 можно изменить состояние с помощью инструкций обработки данных (ADC, ADD, AND, MOV,..), если они изменяют PC.

\subsection{Сопроцессоры}

Архитектура позволяет расширять функциональность, использую сопроцессоры. Может быть адресовано 16 сопроцессоров CP0-CP15, обратиться к ним можно с помощью инструкций MRC, MCR, LDC, STC, CDP. Сопроцессор управления системой (CP15) отвечает за кэш-память и виртуальную память.
Advanced SIMD - набор SIMD (Single Instruction Multiple Data) инструкций, который обеспечивает ускорение для мультимедиа приложений и обработки сигналов. Поддерживает работу с 8-, 16-, 32-, 64-битными элементами целого типа, одинарной точности и с плавающей запятой. VFP(Vector Floating Point) - операции над числами с плавающей запятой в соответствии с ANSI/IEEE-754.




