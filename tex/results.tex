%результаты

Результатом работы стал алгоритм для декодирования инструкций процессора ARM, который использует предварительное индексирование инструкций на основе их масок. Алгоритм может работать с любым имеющимся набором инструкций ARM и гарантирует поиск инструкции за константное время. На основе алгоритме был написан программный модуль декодирования инструкций, который нашел применение в виртуальной машине для платформы ARM. Причем модуль может быть использован как на нативной платформе, так и для эмуляции ARM на x86; модуль кроссплатформенный, однако при запуске на ARM подключаются некоторые низкоуровневые оптимизации. 

Модуль не может исполняться отдельно, он используется как часть общего виртуализационного решения, поэтому определен внешний интерфейс для работы с ним, позволяющий создавать индексированные таблицы поиска для произвольного набора инструкций платформы ARM. Виртуальная машина - это сложная система, состоящая из многих компонентов, взаимодействующих между собой. Достаточно сложно оценить производительность системы в целом, но можно оценить производительность компонентов и сравнить с доступными аналогичными решениями. Так были проведены замеры времени декодирования модуля Qemu, впоследствие в исследовании отталивались от этих данных.

Во время исследования использовались несколько тестовых окружений:

\begin{enumerate}
    \item PandaBoard(OMAP4430), ARM Cortex-A9 1GHz@2, RAM 1Gb;
    \item BeagleBoard
    \item x$86\_64$ IntelCore i5, 2.27GHz@2, RAM 4Gb
\end{enumerate}

