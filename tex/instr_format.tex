
В работе рассматриваются наборы инструкций ARM, Thumb, Thumb2. Все опкоды имеют фиксированный размер - 16 или 32 бита в зависимости от набора. В опкоде закодированы тип инструкции, аргументы и условия выполнения. Инструкция определяется уникальной комбинацией битов, поэтому для определения инструкции по опкоду можно использовать битовую маску.


\textbf{Битовая маска}

Это данные, которые используются для выбора отдельных битов из двоичной строки. В задаче распознавания инструкций битовая маска используется для выделения значащих битов опкода, то есть тех битов, которые определяют тип инструкции. Операция конъюнкции дает логическую единицу, когда и в маске и в опкоде бит выставлен. Для дальнейших действий с маской введем определения.

Пусть есть битовая строка $s$ размера $k$, тогда $Q_0(s)$ - это множество позиций, считая с младших разрядов, в которых стоит 0, а $Q_1(s)$ определяет множество позиций, где стоит 1. Более формально \ref{eq:bit_position}.

\begin{eqnarray} \label{eq:bit_position}
	Q_0(s) = \{p | (s >> p) \& 0x1 = 0,\ 0 \le p < k\} \nonumber \\ 
	Q_1(s) = \{p | (s >> p) \& 0x1 = 1,\ 0 \le p < k\}
\end{eqnarray}

Количество 1-битов в строке $s$ можно определить как $|Q_0(s)|$ - размер множества позиций. Определим значимый размер битовой строки $s$ как разность старшей и младшей позиции, в которых стоят единицы:

\begin{equation}
	S(s) = max(Q_1(s)) - min(Q_1(s)) + 1
\end{equation}

Битовая маска $s$ может однозначно распознать $2^{Q_1(s)}$ входных битовых строк. 

Маска выделяет в опкоде биты, отвечающие за тип инструкции и скрывает аргументы. Например, рассмотрим инструкцию CMP\ref{img:cmp_instr}, ее маска скрывает аргументы инструкции и условие исполнения. Для маски получаем $|Q_0(mask)| = 24,\ |Q_1(mask)| = 8, S(mask) = 24$. За кодирование типа инструкции отвечают биты <27-21> и <7-4>.


Зададим таблицу соответствия инструкций и их масок. Обращаться к i-й инструкции можно через $instr[i]$, к i-й маске - $mask[i]$. Размер таблицы инструкций $ISET\_SIZE$. Тогда алгоритм поиска инструкции по опкоду выглядит так:

\begin{verbatim}
    for (i = 0; i < ISET_SIZE; i++)
        if (input_opcode & mask[i] == instr[i])
            return instr[i];
    return undefined_instr;
\end{verbatim}


\begin{figure}[h!] \label{img:cmp_instr}
    \center{\includegraphics[width=0.9\linewidth]{instr_fmt_cmp}}
    \caption{Инструкция cmp }
    \label{img:fmt_cmp}
\end{figure}
